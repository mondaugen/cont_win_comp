\def\papertitle{On the design and use of once-differentiable high dynamic
resolution atoms for the Distribution Derivative Method%
}
\def\paperauthorA{Nicholas Esterer}
\def\paperauthorB{Philippe Depalle}
\documentclass[plainsections,landscape]{sciposter}


\usepackage{epsfig}
\usepackage{amsmath}
\usepackage{amssymb}
\usepackage{multicol}
\usepackage{tcolorbox}
\usepackage{caption}
%\usepackage{fancybullets}
%\usepackage{amsmath,amssymb,amsfonts,amsthm}
%\usepackage{euscript}
%\usepackage[latin1]{inputenc}
%\usepackage[T1]{fontenc}
%\usepackage{ifpdf}
%
%\usepackage[english]{babel}
%\usepackage{caption}
%\usepackage{subfig} % or can use subcaption package
%\usepackage{color}
%\usepackage{cite}
%\usepackage{float}
%
%\usepackage{hyphenat}
%\hyphenation{spher-oidal}
%%\hyphenation{amp-li-tude}
%%\hyphenation{mod-u-lat-ed}
%
%\usepackage{times}

\newtheorem{Def}{Definition}

%\definecolor{BoxCol}{rgb}{0.9,0.9,0.9}
% uncomment for grey background to \section* boxes 
% for use with default option boxedsections

%\definecolor{BoxCol}{rgb}{0.9,0.9,1}
% uncomment for light blue background to \section* boxes 
% for use with default option boxedsections

%\definecolor{SectionCol}{rgb}{0,0,0.5}
% uncomment for dark blue \section* text 

% Flowchart drawing stuff
\newcommand{\myinnercolsep}{1cm}
\newcommand{\flowfontsize}{\small}
\usepackage{tikz}
\usetikzlibrary{shapes.geometric, arrows.meta, positioning, calc,
backgrounds}
\tikzstyle{process} = [rectangle, minimum width=3cm, minimum height=0cm, text
centered, draw=black,  font=\flowfontsize{}, text width=0.4\columnwidth]
\tikzstyle{emptybox} = []
\tikzstyle{arrow} = [thick,->,-{Latex}]
\tikzstyle{noarrow} = [thick,-,>=stealth]

\title{\papertitle}

% Note: only give author names, not institute
\author{\paperauthorA{} and \paperauthorB{}}
 
% insert correct institute name
\institute{Sound Processing and Control Laboratory (SPCL), McGill University,
Montreal, Canada}

\email{nicholas.esterer@mail.mcgill.ca, philippe.depalle@mcgill.ca}  % shows author email address below institute

\input{ddm_snr_win_comp_defs.txt}
\input{comp_offset_chirp_est_err_defs.txt}

\setlength{\secskip}{1pt}

\ifthenelse{\equal{\orientation}{landscape}}{
  \setlength{\titleskip}{0.01\textheight}
}{
  \setlength{\titleskip}{0.01\textwidth}
}

\newcommand{\imsize}{\columnwidth}
\newcommand{\mytcbtitlewrap}[1]{{\parbox[bottom][1.5cm][c]{\linewidth}{#1}}}

% This is a huge hack
\renewcommand{\rightlogo}[3][0.9]{%
\ifthenelse{\equal{#2}{}}{%
 \renewcommand{\printrightlogo}{%
   \begin{center}
     \hspace*{0.6\textwidth}
   \end{center}
   }
 }
 {\renewcommand{\printrightlogo}{%
      \begin{center}
      \resizebox{#1\textwidth}{!}{\includegraphics{#2}}
      \vspace{4.5cm}
      \resizebox{#1\textwidth}{!}{\includegraphics{#3}}
    \end{center}
 }}
}

\renewcommand{\leftlogo}[2][0.9]{%
\ifthenelse{\equal{#2}{}}{%
 \renewcommand{\printleftlogo}{%
   \begin{center}
     \hspace*{0.6\textwidth}
   \end{center}
   }
 }
 {\renewcommand{\printleftlogo}{%
      \begin{center}
      \resizebox{#1\textwidth}{!}{\includegraphics{#2}}
      \vspace{4.5cm}
    \end{center}
 }}
}

\renewcommand{\footnotesize}{\small}

\tcbset{colbacktitle=white,coltitle=black,adjust text=\vline height
1cm,valign=center,toptitle=2mm,bottomtitle=2mm,fonttitle=\bfseries}

\setmargins[3cm]
\begin{document}

\leftlogo[1.0]{{DAFx17_logo}.pdf}
\rightlogo[1.0]{{cirmmt_logo}.pdf}{{nserc_logo}.pdf}
%\rightlogo[1.0]{{nserc_logo}.pdf}
\renewcommand{\footlogo}{%
    \resizebox{\logowidth}{!}{\includegraphics{{nserc_logo}.pdf}\hspace{1cm}\includegraphics{{cirmmt_logo}.pdf}}%
}

%define conference poster is presented at (appears as footer)

\conference{{\em{\small{The 20$^{\text{\itshape th}}$ International Conference
on Digital Audio Effects (DAFx-17), Edinburgh, UK, September 5--9, 2017, \textbf{pp.
208--214}.}}}}

%\LEFTSIDEfootlogo  
% Uncomment to put footer logo on left side, and 
% conference name on right side of footer

% Some examples of caption control (remove % to check result)

%\renewcommand{\algorithmname}{Algoritme} % for Dutch

\renewcommand{\mastercapstartstyle}[1]{\small{\textbf{#1}}}

\renewcommand{\mastercapbodystyle}[1]{\small{}#1}
%\renewcommand{\algcapstartstyle}[1]{\textsc{\textbf{#1}}}
%\renewcommand{\algcapbodystyle}{\bfseries}
%\renewcommand{\thealgorithm}{\Roman{algorithm}}

\maketitle

%\newpage

%\newpage

%%% Begin of Multicols-Enviroment
\setlength{\columnseprule}{0pt}
\begin{multicols}{3}

%%% Abstract
\begin{tcolorbox}[title=\mytcbtitlewrap{Abstract}]
\setlength{\columnsep}{\myinnercolsep}
\begin{multicols}{2}
    \textit{%
        The \textbf{accuracy of the Distribution Derivative Method (DDM)
    \cite{betser2009sinusoidal}} is evaluated on mixtures of \textbf{chirp
    signals}. It is
    shown that accurate estimation can be obtained when the sets of atoms for
    which the inner product is large are disjoint.  This amounts to \textbf{designing
    atoms} with windows whose Fourier transform exhibits \textbf{low side-lobes but which
    are once-differentiable in the time-domain}. A \textbf{technique for designing
    once-differentiable approximations} to windows is presented and the \textbf{accuracy
    of these windows in estimating the parameters} of sinusoidal chirps in
    mixture \textbf{is evaluated}.}
\end{multicols}
\end{tcolorbox}

%%% Introduction
    \begin{tcolorbox}[title=\mytcbtitlewrap{Introduction}]
\setlength{\columnsep}{\myinnercolsep}
\begin{multicols}{2}
\label{sec:intro}
    \begin{itemize}
        \item{Recently, there has
            been interest in using \textbf{higher-order phase functions} for the argument of sinusoidal
    functions \cite{xuepiecewise} as the estimation of
their parameters has been made possible by a set of \textbf{techniques 
            using signal derivatives} \cite{hamilton2012unified}. The DDM is one
            such technique.}

        \item{For comparison with previous research on the DDM, we consider the estimation -- via the DDM -- of the parameters of a
            \textbf{parabolic phase function} for which the imaginary parts of the parameters
            represent the \textbf{initial phase, frequency and frequency slope}. The real parts
    of the parameters represent the \textbf{initial amplitude, amplitude slope and
            amplitude curvature}.}

        \vspace{1cm}

        \item{The DDM requires a set of atoms that are \textbf{once-differentiable everywhere in
            the time-domain} from which the parameters are estimated. We
    show that on a \textbf{mixture of sinusoids, accurate estimation for each
            sinusoid}
    can be obtained under a single-component
            assumption when components have roughly \textbf{disjoint sets of atoms} for which their
inner products take on large values.}

\item{\textbf{Encouraging disjointedness} amounts to designing atoms whose Fourier
    transform exhibits \textbf{low side-lobes} under the additional constraint that
            the atoms be \textbf{everywhere once-differentiable}. We show how to \textbf{design atoms
            with this latter property} based on the shapes of ideal windows.}
    \end{itemize}
\end{multicols}
\end{tcolorbox}

\addtocounter{footnote}{1}\footnotetext{\label{fn:mathdefs}
$\mathcal{T}^{\alpha} : (\mathcal{T}^{\alpha}x)(t) = t^{\alpha}x(t)$,
$\left\langle x , \psi \right\rangle = 
\int_{-\infty}^{\infty}x(t)\overline{\psi}(t)dt$
}
\addtocounter{footnote}{1}\footnotetext{\label{fn:optnote}
The notation $x^{\ast}$ will mean the value of the argument $x$ maximizing or minimizing some
function.
}%

\begin{tcolorbox}[title=\mytcbtitlewrap{Estimating the
    \lowercase{$a_{p,q}$} of $P$ components}]
\setlength{\columnsep}{\myinnercolsep}
\begin{multicols}{2}
The model for signal $x$ is
%
\begin{equation}
    \label{eq:polyphaseexpmix}
    x(t) = \sum_{p=1}^{P} x_{p}(t) + \eta(t)
\end{equation}
%
with $ x_{p}(t) = \exp(a_{p,0} + \sum_{q=1}^{Q} a_{p,q} t^q) $

and $\eta$ Gaussian-distributed white noise. 

Applying the DDM to a mixture of $P$ components, we
    obtain\footnotemark[1]%\ref{fn:mathdefs}]
%
\begin{multline}
    \label{eq:mixest}
    \sum_{p=1}^{P} \left(
    \sum_{q=1}^{Q} q a_{p,q} 
    \left\langle \mathcal{T}^{q-1} x_{p} , \overline{\psi} \right\rangle
    + \left\langle x_{p}, \frac{d\overline{\psi}}{dt} \right\rangle \right)
    \\
    = 0
\end{multline}
%
When $\left\langle \mathcal{T}^{q-1} x_{p} , \overline{\psi}_{r}
\right\rangle$ and $\left\langle x_{p}, \frac{d\overline{\psi_{r}}}{dt} \right\rangle$
are small for all but $p = p^{\ast}$ and a subset of $R$ atoms, we can simply
    estimate the parameters $a_{p^{\ast},q}$ using%
    \footnotemark[2]%[\ref{fn:optnote}]
\begin{multline}
    \sum_{q=1}^{Q} q a_{{p^{\ast}},q} 
    \left\langle \mathcal{T}^{q-1} x_{p^{\ast}} , \overline{\psi}_{r} \right\rangle
    = -\left\langle x_{p^{\ast}}, \frac{d\overline{\psi}_{r}}{dn}
    \right\rangle\\
    \text{for } 1 \leq r \leq R\text{.}
\end{multline}
\end{multicols}
\end{tcolorbox}

\begin{tcolorbox}[title=\mytcbtitlewrap{Designing the $\psi_{r}$}]
\setlength{\columnsep}{\myinnercolsep}
\begin{multicols}{2}
\label{sec:designingatoms}
%
Canonically, the atoms $\psi_{\omega}(t)$ 
    are the \textbf{products} of the elements of the \textbf{Fourier basis} and a 
    \textbf{window $w$ that is once differentiable}, i.e.,
    $\psi_{\omega}(t) = w(t) \exp(-j \omega t)$.

    Defining $N$ \textbf{as the length of the window} in samples, angular frequency at bin $r$ as $\omega_{r} = 2
\pi \frac{r}{N}$, and $T$ the sample period in seconds, the approximate
    \textbf{inner product} $\left\langle x , \psi_{\omega}
    \right\rangle$ equals:
\begin{multline}
    \label{eq:approxinnerprod}
    \sum_{n=0}^{N-1} x(Tn) w(Tn) \exp(-2 \pi j r \frac{n}{N}) 
\end{multline}
i.e., the definition of the DFT of a windowed signal.
\columnbreak

Therefore \textbf{choosing $\psi$ amounts to
    designing a Fourier window with minimal side-lobes}, under the \textbf{constraint
    that the window be differentiable in $t$ and finite}.
\end{multicols}
\end{tcolorbox}

\begin{tcolorbox}[title=\mytcbtitlewrap{Differentiable approximations to windows}]
%
\begin{figure}
\begin{center}
    {\resizebox{\imsize}{!}{\includegraphics{{search_dpw_bw_m}.eps}}}
\end{center}
    \caption{\label{fig:dpw} \textbf{Comparing the main-lobe and asymptotic power
    spectrum} characteristics of the continuous 4-term Nuttall window, the digital
prolate window with $W=0.008$, and the \textbf{continuous approximation to the digital
    prolate window}. The falloff of the
    Nuttall and Prolate Approximation is \textbf{18 dB per octave} because they are both
    \textbf{once-differentiable with respect to time}.}
\end{figure}
%
\setlength{\columnsep}{\myinnercolsep}
\begin{multicols}{2}
A differentiable approximation to a symmetrical window of length $N$ samples can be designed using a \textbf{linear
    combination of $M$ harmonically related cosines} \cite{harris1978use} \cite{rabiner1970approach} 
\begin{equation}
    \tilde{w}(n) = \sum_{m=0}^{M-1} b_{m} \cos (2 \pi m \frac{n}{N})
\mathcal{R}(\frac{n}{N})
\end{equation}
where $\mathcal{R}$ is the \textit{rectangle function}.
    
$\tilde{w}(n)$ is
not differentiable
at $n = \frac{\pm N}{2}$, unless
$\sum_{m=0}^{M-1} b_{m} \cos ( \pm \pi m ) = 0$.

Therefore, we choose the $b_{m}$ 
minimizing the window $\tilde{w}$'s approximation error to $w$ while
    \textbf{maintaining continuity}, finding the solution $\{ b^{\ast}_{m} \}$ to the
mathematical program
\begin{equation*}
    \label{eq:searchcontwinprogram}
    \text{minimize}
    \sum_{n=0}^{N-1} ( w(n) 
        - \sum_{m=0}^{M-1} b_{m} \cos(2 \pi m \frac{n}{N}))^{2}
\end{equation*}
\begin{equation}
    \text{subject to} \\
    \sum_{m=0}^{M-1} b_{m} \cos(\pi m ) = 0
\end{equation}

    Using this technique, we designed a \textbf{5-term continuous approximation} to a digital \textbf{prolate
    spheroidal window} of length $N=512$ and design parameter $W=0.008$ \cite{verma1996digital}. The coefficients
are described in Table~\ref{tab:contprolate} and a comparison with other
windows is shown in Figure~\ref{fig:dpw}.

\begin{table}[]
    \caption{The coefficients of the once-differentiable approximation to a digital prolate
    window.
    \label{tab:contprolate}}
    \begin{center}
        \begin{tabular}{c c c}
            $b_0$ & $ = $ & 3.128 $\times 10^{-1}$ \\
            $b_1$ & $ = $ & 4.655 $\times 10^{-1}$ \\
            $b_2$ & $ = $ & 1.851 $\times 10^{-1}$ \\
            $b_3$ & $ = $ & 3.446 $\times 10^{-2}$ \\
            $b_4$ & $ = $ & 2.071 $\times 10^{-3}$ 
        \end{tabular}
    \end{center}
\end{table}%
\end{multicols}
\end{tcolorbox}

\begin{tcolorbox}[title=\mytcbtitlewrap{The performance of improved windows on a
    mixture of 2 sinusoids}]
%    \setlength{\floatsep}{-20pt}
    \begin{center}
\begin{figure}
\begin{center}
    {\resizebox{\imsize}{!}{\includegraphics{{comp_offset_chirp_est_err_params}.eps}}}
\end{center}
\caption{\label{fig:mixwincomp} The mean squared estimation error for each
    parameter \textbf{in an analysis of two components in mixture}. A set of
    \textbf{$K_{2}=\Koffset{}$ chirps} was synthesized and \textbf{each unique
    pair} used for
    maximum bin differences $0 \leq d < \Doffset{}$, with $d$ varied in
    $\Delta_{d} = \Dstep{}$ bin increments. The \textbf{signal power ratio}
    between components is indicated with \textbf{colours} and the corresponding ratio in decibels is indicated in the
    plot legend (the power ratios
    $\mathcal{S}$ tested were \textbf{0 dB and -30 dB}). The names indicate the windows used to generate the atoms for
    estimation: \textbf{\textit{N3} and \textit{N4} are the 3- and 4-cosine-term
    continuous Nuttall} windows, \textbf{\textit{H} is the Hann window}, and \textbf{\textit{P5}
    is the continuous 5-cosine-term approximation to a digital prolate window}.}
\end{figure}

\begin{figure}
    \centering
    \begin{minipage}[l]{0.65\columnwidth}
            \begin{tikzpicture}[node distance=0.25cm, background rectangle/.style= {draw=white,fill=white}, show background rectangle]
%                \begin{scope}[shift={(-5,0)}]
                \node (start) [process] at (0,0) {Synthesize $K_{2}$
                single-component signals.};
                \node (dumstart) [emptybox, right = of start] {};
                \node (modulate) [process, below = of start] {Modulate so that
                peak bin is at frequency 0 for all signals.};
                \node (setd) [process, below = of modulate] {Set $d=0$.};
                \node (choosepair) [process, below = of setd] {Choose
                    $K_{2}(K_{2}-1)$ pairs of signals.};
                \node (dum1) [emptybox, right = of choosepair] {};
                \node (scaleone) [process, below = of choosepair] {Scale one in
                each pair to give desired power from $\mathcal{S}$ and modulate to peak bin
                according to $d$.};
                \node (dumscaleone) [emptybox, right = of scaleone] {};
                \node (addpair) [process, right = of dumstart] {Add each signal
                pair together.};
                \node (tryestimate1) [process, below = of addpair] {For each
                pair, try
                estimating parameters of unmodulated component with atoms at
                bins $\{-1,0,1\}$.};
                \node (tryestimate2) [process, below = of tryestimate1] {For
                each pair, try
                estimating parameters of modulated component with atoms at bins
                $\{d-1,d,d+1\}$.};
                \node (sumerrs) [process, below = of tryestimate2] {Sum estimation errors of
                each parameter and divide by $K_{2}(K_{2}-1)$.};
                \node (incd) [process, below = of sumerrs] {Increment $d$ by
                $\Delta_{d}$.};
                \node (dum2) [emptybox, left = of incd, font=\flowfontsize{},
                anchor=east] {If $d < D$.};
                \node (dum3) [emptybox, below = of incd] {};
                \node (dum3b) [emptybox, left = of scaleone] {};
                \node (dum4) [emptybox] at (dum3b |- dum3) {};
                \node (dum4text) [emptybox, text width=0.3\columnwidth, above right, 
                font=\flowfontsize{}] at (dum4.center) {If signal power ratios in $\mathcal{S}$ remaining to be
                evaluated.};
%                \node let \p{dum3b}=(dum3b), \p{dum3}=(dum3) in (dum4) [emptybox] at (\x{dum3b},\y{dum3}) {};
                \node (dum5) [emptybox, left = of setd] {};

                \draw [arrow] (start) -- (modulate);
                \draw [arrow] (modulate) -- (setd);
                \draw [arrow] (setd) -- (choosepair);
                \draw [arrow] (choosepair) -- (scaleone);
                \draw [arrow] (scaleone) -- (dumscaleone.center) --
                (dumstart.center) -- (addpair);
                \draw [arrow] (addpair) -- (tryestimate1);
                \draw [arrow] (tryestimate1) -- (tryestimate2);
                \draw [arrow] (tryestimate2) -- (sumerrs);
                \draw [arrow] (sumerrs) -- (incd);

                \draw [arrow] (incd) -- (dum2.east) -- 
                    (dum1.east) -- (choosepair);
                %\node [emptybox,text width=4cm,minimum width=3cm,left = of dum2,
                %align = flush right]
                %{If $d < D$.};

                \draw [arrow] (incd) -- (dum3.center) -- 
                (dum4.center) -- 
                 (dum5.center) -- (setd);
%            \end{scope}
    \end{tikzpicture}
    \end{minipage}
    \begin{minipage}[r]{0.3\columnwidth}
        \caption{The \textbf{evaluation procedure for 2-component
        signals}.\label{fig:2cevalflowchart}
We synthesized signals using
Eq.~\ref{eq:polyphaseexpmix} with $P=2$, $Q=2$ and parameters chosen from the
        \textbf{uniform distributions specified in \cite{betser2009sinusoidal}}.
        No noise was added to the mixture. We then evaluated
the parameter estimation accuracy at \textbf{various differences between local maxima in
        the mixture's power-spectrum}.}
    \end{minipage}
\end{figure}
\end{center}
\end{tcolorbox}

\bibliographystyle{IEEEbib}
%\nocite{*}
%\bibliography{dafx-poster} % requires file DAFx17_tmpl.bib
\bibliography{paper}
\end{multicols}

\end{document}

