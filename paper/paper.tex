% Template LaTeX file for DAFx-17 papers
%
% To generate the correct references using BibTeX, run
%     latex, bibtex, latex, latex
%
% 1) Please compile using latex or pdflatex.
% 2) If using pdflatex, you need your figures in a file format other than eps! e.g. png or jpg is working
% 3) Please use "paperftitle" and "pdfauthor" definitions below

%------------------------------------------------------------------------------------------
%  !  !  !  !  !  !  !  !  !  !  !  ! user defined variables  !  !  !  !  !  !  !  !  !  !  !  !  !  !
% Please use these commands to define title and author(s) of the paper:
\def\papertitle{Improved Atoms for the Distribution Derivative Method}
\def\paperauthorA{Nicholas Esterer}
\def\paperauthorB{Philippe Depalle}

% Authors' affiliations have to be set below

%------------------------------------------------------------------------------------------
\documentclass[twoside,a4paper]{article}
\usepackage{dafx_17}
\usepackage{amsmath,amssymb,amsfonts,amsthm}
\usepackage{euscript}
\usepackage[latin1]{inputenc}
\usepackage[T1]{fontenc}
\usepackage{ifpdf}

\usepackage[english]{babel}
\usepackage{caption}
\usepackage{subfig} % or can use subcaption package
\usepackage{color}

\setcounter{page}{1}
\ninept

\usepackage{times}
% Saves a lot of ouptut space in PDF... after conversion with the distiller
% Delete if you cannot get PS fonts working on your system.

% pdf-tex settings: detect automatically if run by latex or pdflatex
\newif\ifpdf
\ifx\pdfoutput\relax
\else
   \ifcase\pdfoutput
      \pdffalse
   \else
      \pdftrue
\fi

\ifpdf % compiling with pdflatex
  \usepackage[pdftex,
    pdftitle={\papertitle},
    pdfauthor={\paperauthorA, \paperauthorB},
    colorlinks=false, % links are activated as colror boxes instead of color text
    bookmarksnumbered, % use section numbers with bookmarks
    pdfstartview=XYZ % start with zoom=100% instead of full screen; especially useful if working with a big screen :-)
  ]{hyperref}
  \pdfcompresslevel=9
  \usepackage[pdftex]{graphicx}
  \usepackage[figure,table]{hypcap}
\else % compiling with latex
  \usepackage[dvips]{epsfig,graphicx}
  \usepackage[dvips,
    colorlinks=false, % no color links
    bookmarksnumbered, % use section numbers with bookmarks
    pdfstartview=XYZ % start with zoom=100% instead of full screen
  ]{hyperref}
  % hyperrefs are active in the pdf file after conversion
  \usepackage[figure,table]{hypcap}
\fi

\title{\papertitle}

%-------------SINGLE-AUTHOR HEADER STARTS (uncomment below if your paper has a single author)-----------------------
%\affiliation{
%\paperauthorA \,\sthanks{Thanks to the predecessors for the templates}}
%{\href{http://www.acoustics.ed.ac.uk}{Acoustics and Audio Group,} \\ University of Edinburgh\\ Edinburgh, UK\\
%{\tt \href{mailto:dafx17@ed.ac.uk}{dafx17@ed.ac.uk}}
%}
%-----------------------------------SINGLE-AUTHOR HEADER ENDS------------------------------------------------------

%---------------TWO-AUTHOR HEADER STARTS (uncomment below if your paper has two authors)-----------------------
\twoaffiliations{
\paperauthorA \,\sthanks{}}
{\href{http://www.cirmmt.org/}{https://mt.music.mcgill.ca/spcl/home} %
\\ McGill University %
\\ Montreal, Quebec, Canada \\
{\tt \href{mailto:nicholas.esterer@mail.mcgill.ca}{nicholas.esterer@mail.mcgill.ca}}
}
{\paperauthorB \,\sthanks{}}
{\href{http://www.cirmmt.org/}{https://mt.music.mcgill.ca/spcl/home} %
\\ McGill University %
\\ Montreal, Quebec, Canada \\
{\tt %
\href{mailto:philippe.depalle@music.mcgill.ca}{philippe.depalle@music.mcgill.ca}}
}

\begin{document}
% more pdf-tex settings:
\ifpdf % used graphic file format for pdflatex
  \DeclareGraphicsExtensions{.png,.jpg,.pdf}
\else  % used graphic file format for latex
  \DeclareGraphicsExtensions{.eps}
\fi

\maketitle

\begin{abstract}
TODO
\end{abstract}

\section{Introduction}
\label{sec:intro}
%Additive synthesis using a sum of sinusoids plus noise is a powerful model for
%representing audio, allowing for the easy implementation of many manipulations
%such as time-stretching and transposition. In its most basic form, peaks in the
%short-time spectrum are identified as representing sinusoids and connected based
%on some criteria. As the phase evolution of the sinusoid is assumed linear over
%the analysis frame, only the phase and frequency of the sinusoids at these
%analysis points are used to fit a plausible phase function. Recently, there has
%been interest in using higher order phase functions as the estimation of their
%parameters has been made possible by a new set of techniques of only moderate
%computational complexity using signal derivatives. Theoretically, the estimation
%of the parameters of a polynomial phase function of arbitrary order make it
%possible to approximate any analytical phase function to desired precision.
%Empirically, the use of higher order phase models allow for higher quality
%manipulations.
%
%The technique we will consider in this paper is the Distribution Derivative
%Method for its convenience: the technique does not require computing derivatives
%of the signal to be analysed. It does, however, require a
%$p$ times differentiable analysis window where $p+1$ is the order of the
%phase polynomial to be estimated.
%
%In the few reports on the technique, the window used is the Hann window. This
%window is many-times differentiable but 
%
%Complex polynomial phase models whose estimation is 
\section{The polynomial phase sinusoid}
Recently techniques for estimating the parameters of complex polynomial phase sinusoids,
i.e., signals of the form
\[
    x(t) = \exp(a_0 + \sum_{j=1}^{Q} a_j t^j)
\]
with $a_{i} \in \Complex$, have been developed. Many equivalent techniques have
been developed for estimating the $a_j$. In this paper we compare with the
Distribution Derivative Method (DDM) only, without loss of generality.

\section{The estimation of $a_j$}
We will now show briefly how the DDM can be used to estimate the $a_j$. Based on
the theory of distributions (Laurent Schwartz), the DDM
makes use of ``test
functions'' or atoms $\psi$. These atoms must be once differentiable with respect to
$t$ and be non-zero only on a finite interval $[-\frac{L_{t}}{2},\frac{L_{t}}{2}]$. First, we define
%
\begin{equation}
    \label{eq:ddm:inner:prod:def}
    \left\langle x , \psi \right\rangle = 
    \int_{-\infty}^{\infty}x(t)\overline{\psi}(t)dt
\end{equation}
%
and the operator 
%
\[
\mathcal{T}^{\alpha} : (\mathcal{T}^{\alpha}x)(t) = t^{\alpha}x(t)
\]
%
Consider the weighted signal
%
\[
    f(t) = x(t) \overline{\psi}(t)
\]
%
differentiating with respect to $t$ we obtain
%
\begin{multline}
    \label{eq:ddm:weighted:sig:derivative}
    \frac{df(t)}{dt} = 
    \frac{dx}{dt}(t)\overline{\psi}(t)
    + x(t)\frac{d\overline{\psi}}{dt}(t) \\
    = \left( \sum_{j=1}^{Q} j a_j t^{j-1} \right) x(t)\overline{\psi}(t)
    + x(t)\frac{d\overline{\psi}}{dt}(t)
\end{multline}
%
Because $\psi$ is zero outside of the interval $[-\frac{L_{t}}{2},\frac{L_{t}}{2}]$, integrating
$\frac{df(t)}{dt}$ we obtain
%
\begin{multline*}
    \int_{-\infty}^{\infty}\frac{df(t)}{dt}dt \\
    = \sum_{j=1}^{Q} j a_j \int_{-\frac{L_{t}}{2}}^{\frac{L_{t}}{2}} t^{j-1} x(t) \overline{\psi}(t) dt
    + \left\langle x, \frac{d\overline{\psi}}{dt} \right\rangle = 0
\end{multline*}
%
or, using the operator $\mathcal{T}^{\alpha}$,
%
\[ 
    \sum_{j=1}^{Q} j a_j 
    \left\langle \mathcal{T}^{j-1} x , \overline{\psi} \right\rangle
    = -\left\langle x, \frac{d\overline{\psi}}{dt} \right\rangle
\]
%

From this we can see that to estimate the coefficients $a_j$, $ 1 \leq q \leq Q
$ we simply need $R$ atoms with $R \geq Q$ to solve the linear system of
equations
\begin{equation}
    \label{eq:ddmsyseq}
    \sum_{j=1}^{Q} j a_j 
    \left\langle \mathcal{T}^{j-1} x , \overline{\psi}_{r} \right\rangle
    = -\left\langle x, \frac{d\overline{\psi}_{r}}{dn} \right\rangle
\end{equation}
for $1 \leq r \leq R$.

\section{The estimation of the $a_{i,j}$ of $P$ components}

Although this technique can be exteded to describe a single
sinusoid of arbitrary complexity simply by increasing $Q$, it is still of
interest to consider a signals featuring a sum of $P$ such components, i.e.,
%
\[
    s(t) = \sum_{i=1}^{P} x_{i}(t)
\]
%
with
%
\[
    x_{i}(t) = \exp(a_{i,0} + \sum_{j=1}^{Q} a_{i,j} t^j)
\]
%
In situations where multiple signal sources are known to exist (e.g., recordings
of multiple speakers or performers) such a model is plausible.

We examine how this mixture model influences the estimation of the $a_{i,j}$.
Consider a mixture of $P$ components.
If we define the weighted signal sum
%
\[
    g(t) = \sum_{i=1}^{P} x_{i}(t) \overline{\psi}(t) = \sum_{i=1}^{P} f_{i}(t)
\]
%
and substitute $g$ for $f$ in \ref{eq:ddm:weighted:sig:derivative} we obtain
%
\begin{multline}
    \sum_{i=1}{P} \int_{-\frac{L_{t}}{2}}^{\frac{L_{t}}{2}} \frac{df_{i}}{dt}(t) =
    0
    \\ = 
    \sum_{i=1}^{P}
    \sum_{j=1}^{Q} j a_{i,j} 
    \left\langle \mathcal{T}^{j-1} x_i , \overline{\psi} \right\rangle
     + \left\langle x_i, \frac{d\overline{\psi}}{dt} \right\rangle
\end{multline}
%
From this we see if $\left\langle \mathcal{T}^{j-1} x_i , \overline{\psi}_{r}
\right\rangle$ and $\left\langle x_i, \frac{d\overline{\psi_{r}}}{dt} \right\rangle$
are small for all but $i = i^{\ast}$ and a subset of $R$ atoms, we
can simply estimate the parameters $a_{i^{\ast},j}$ using
\[
    \sum_{j=1}^{Q} j a_{i,j} 
    \left\langle \mathcal{T}^{j-1} x_i , \overline{\psi}_{r} \right\rangle
    = -\left\langle x_i, \frac{d\overline{\psi}_{r}}{dn} \right\rangle
\]
for $1 \leq r \leq R$.

\section{Designing the $\psi_{r}$}

In practice, an approximation of \ref{eq:ddm:inner:prod:def} is evaluated using
the discrete Fourier transform (DFT) on a signal $x$ that is properly sampled
and so can be evaluated at a finite number of points $nT$ with $n \in [0,N-1]$ and
$T$ the sample period in seconds. In this way, the atoms $\psi_{\omega}(t)$ chosen
are the products of the elements of the Fourier basis and an appropriately
chosen window $w$ that is once differentiable and finite, i.e.,
%
\[
    \psi_{\omega}(t) = w(t) \exp(-\sqrt{-1} \omega t)
\]
%
Defining $N = \frac{L_{t}}{T}$ and $\omega = 2
\pi \frac{r}{N}$, the approximate
inner product is then
%
\[
    \left\langle x , \psi_{\omega} \right\rangle \approx 
    \frac{1}{N} \sum_{n=0}^{N-1} x(Tn) w(Tn) \exp(-2 \pi \sqrt{-1} r \frac{n}{N}) 
\]
%
i.e., the definition of the DFT of a windowed signal.
\footnote{%
    Notice however that this is an approximation of the inner product and should
    not be interpreted as yielding the Fourier series coefficients of a properly
    sampled signal $x$ periodic in $L_{t}$. This means that other
    evaluations of the inner product that yield more accurate results are
    possible. For example, the analytic solution is possible if $x$ is assumed
    zero outside of $[-\frac{L_{t}}{2},\frac{L_{t}}{2}]$ (the $\psi$ are in
    general analytic).  In this case the samples of $x$ are convolved with the
    appropriate interpolating sinc functions and the integral of this function's
    product with $\psi$ is evaluated.
}%
The DFT is readily interpreted as a bank of bandpass filters centred at
normalized frequencies $\frac{r}{N}$ and with frequency response described by
the Fourier transform of modulated $w$. Therefore choosing $\psi$ amounts to a
filter design problem under the constraints that the impulse response of the
filter be differentiable in $t$ and finite. To minimize the influence of
all but one component, granted the components's energy concentrations are
sufficiently separated in frequency, we desire impulse responses whose magnitude
response gives maximum
out-of-band rejection or equivalently, windows whose Fourier transform exhibits
the lowest side-lobes.

In all the publications these authors reviewed on the DDM, the window used was
the Hann window which is once-differentiable everywhere in the time-domain.
Nuttall has designed windows with lower side-lobes than the canonical Hann
window which are everywhere at least once-differentiable. It is also possible to
design approximations to aribtrary symmetrical window functions using
harmonically related cosines, as is discussed in the following section.

\section{Differentiable approximations to windows}

An differentiable approximation to a symmetrical window can be designed in a
straight-forward way. (Harris) and (McGonnegal) have shown how to design optimal
windows of length $N$ samples using a linear combination of $M$ harmonically
related cosines
\begin{equation}
    w(n) = \sum_{k=0}^{M-1} b_{k} \cos (2 \pi k \frac{n}{N})
\mathcal{R}(\frac{n}{N})
\end{equation}
where $\mathcal{R}$ is the \textit{rectangle function}. This function is discontinous
at $n = \frac{\pm N}{2}$, and therefore not differentiable there, unless
\[
\sum_{k=0}^{M-1} b_{k} \cos ( \pm \pi k ) = 0
\]

We choose the $b_{k}$ so that the window $w$'s squared approximation error to
$v$ is minimized
while having $w(\frac{\pm N}{2}) = 0$, i.e. we find the solution $\{
b^{\ast}_{k} \}$ to the mathematical program
\begin{equation}
    \text{minimize}
    \sum_{n=0}^{N-1} ( v(n) 
        - \sum_{k=0}^{M-1} b_{k} \cos(2 \pi k \frac{n}{N}))^{2}
\end{equation}
\[
    \text{subject to} \\
    \sum_{k=0}^{M-1} b_{k} \cos(\pi k ) = 0
\]
which can be solved using standard numerical linear algebra routines (Goulob and
Van Loan p 585).

\section{Approximating digital prolate spheroidal windows}

Digital prolate spheroidal windows are a parametric approximation to functions
proposed by Slepian et al. maximimzing the energy in a given bandwidth. These can be
tuned to have extremely low side-lobes, at the expense of main-lobe width.
Differentiation of these window functions may be possible but is not as
straightforward as differentiation of the sum-of-cosine windows. Furthermore,
the windows do not generally have end-points equal to 0. In the following we
will demonstrate how to approximate one of these windows.

(Verna, Bilbao et al.) showed how to construct prolate spheroidal windows under
parmeters $N$, the window length in samples, and $W$. We chose $N$ based on the
window length chosen in (Betser) for ease of comparison. $W^{\ast}$ was chosen
by synthesizing windows with $W$ ranging between $0.01$ and $0.99$ at a resolution
of $0.01$. The window with the lowest height of the highest sidelobe was obtained
with $W=0.77$. It is shown in (figure). 
mainlobe width.

The approximation method yields a window virtually identical to the digital
prolate spheroidal window: in our realisation, the approximation error was
everywhere less than $10^{-13}$.

% TODO: Plot these windows

\section{Conclusions}

\section{Acknowledgments}

%\newpage
\nocite{*}
\bibliographystyle{IEEEbib}
%\bibliography{paper} % requires file DAFx17_tmpl.bib

\section{Appendix}

\end{document}
