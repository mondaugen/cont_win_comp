% Template LaTeX file for DAFx-17 papers
%
% To generate the correct references using BibTeX, run
%     latex, bibtex, latex, latex
%
% 1) Please compile using latex or pdflatex.
% 2) If using pdflatex, you need your figures in a file format other than eps! e.g. png or jpg is working
% 3) Please use "paperftitle" and "pdfauthor" definitions below

%------------------------------------------------------------------------------------------
%  !  !  !  !  !  !  !  !  !  !  !  ! user defined variables  !  !  !  !  !  !  !  !  !  !  !  !  !  !
% Please use these commands to define title and author(s) of the paper:
\def\papertitle{Improved Atoms for the Distribution Derivative Method}
\def\paperauthorA{Nicholas Esterer}
\def\paperauthorB{Philippe Depalle}

% Authors' affiliations have to be set below

%------------------------------------------------------------------------------------------
\documentclass[twoside,a4paper]{article}
\usepackage{dafx_17}
\usepackage{amsmath,amssymb,amsfonts,amsthm}
\usepackage{euscript}
\usepackage[latin1]{inputenc}
\usepackage[T1]{fontenc}
\usepackage{ifpdf}

\usepackage[english]{babel}
\usepackage{caption}
\usepackage{subfig} % or can use subcaption package
\usepackage{color}

\setcounter{page}{1}
\ninept

\usepackage{hyphenat}
\hyphenation{spher-oidal}
%\hyphenation{amp-li-tude}
%\hyphenation{mod-u-lat-ed}

\usepackage{times}
% Saves a lot of ouptut space in PDF... after conversion with the distiller
% Delete if you cannot get PS fonts working on your system.

% pdf-tex settings: detect automatically if run by latex or pdflatex
\newif\ifpdf
\ifx\pdfoutput\relax
\else
   \ifcase\pdfoutput
      \pdffalse
   \else
      \pdftrue
\fi

\ifpdf % compiling with pdflatex
  \usepackage[pdftex,
    pdftitle={\papertitle},
    pdfauthor={\paperauthorA, \paperauthorB},
    colorlinks=false, % links are activated as colror boxes instead of color text
    bookmarksnumbered, % use section numbers with bookmarks
    pdfstartview=XYZ % start with zoom=100% instead of full screen; especially useful if working with a big screen :-)
  ]{hyperref}
  \pdfcompresslevel=9
  \usepackage[pdftex]{graphicx}
  \usepackage[figure,table]{hypcap}
\else % compiling with latex
  \usepackage[dvips]{epsfig,graphicx}
  \usepackage[dvips,
    colorlinks=false, % no color links
    bookmarksnumbered, % use section numbers with bookmarks
    pdfstartview=XYZ % start with zoom=100% instead of full screen
  ]{hyperref}
  % hyperrefs are active in the pdf file after conversion
  \usepackage[figure,table]{hypcap}
\fi

\title{\papertitle}

%-------------SINGLE-AUTHOR HEADER STARTS (uncomment below if your paper has a single author)-----------------------
%\affiliation{
%\paperauthorA \,\sthanks{Thanks to the predecessors for the templates}}
%{\href{http://www.acoustics.ed.ac.uk}{Acoustics and Audio Group,} \\ University of Edinburgh\\ Edinburgh, UK\\
%{\tt \href{mailto:dafx17@ed.ac.uk}{dafx17@ed.ac.uk}}
%}
%-----------------------------------SINGLE-AUTHOR HEADER ENDS------------------------------------------------------

%---------------TWO-AUTHOR HEADER STARTS (uncomment below if your paper has two authors)-----------------------
\twoaffiliations{
\paperauthorA \,\sthanks{}}
{\href{http://www.cirmmt.org/}{https://mt.music.mcgill.ca/spcl/home} %
\\ McGill University %
\\ Montreal, Quebec, Canada \\
{\tt \href{mailto:nicholas.esterer@mail.mcgill.ca}{nicholas.esterer@mail.mcgill.ca}}
}
{\paperauthorB \,\sthanks{}}
{\href{http://www.cirmmt.org/}{https://mt.music.mcgill.ca/spcl/home} %
\\ McGill University %
\\ Montreal, Quebec, Canada \\
{\tt %
\href{mailto:philippe.depalle@music.mcgill.ca}{philippe.depalle@music.mcgill.ca}}
}

\input{ddm_snr_win_comp_defs.txt}
\input{comp_offset_chirp_est_err_defs.txt}

\begin{document}
% more pdf-tex settings:
\ifpdf % used graphic file format for pdflatex
  \DeclareGraphicsExtensions{.png,.jpg,.pdf}
\else  % used graphic file format for latex
  \DeclareGraphicsExtensions{.eps}
\fi

\maketitle

\begin{abstract}
    The accuracy of the Distribution Derivative Method (DDM)
    \cite{betser2009sinusoidal} is evaluated on mixtures of chirp signals. It is
    shown that accurate estimation can be obtained when the sets of atoms for
    which the inner product is large are disjoint.  This amounts to designing
    atoms with windows whose Fourier transform exhibits low sidelobes but which
    are once-differentiable in the time-domain. A technique for designing
    once-differentiable approximations to windows is presented and the accuracy
    of these windows in estimating the parameters of chirps in mixture is
    evaluated.
\end{abstract}


\section{Introduction}
\label{sec:intro}
Additive synthesis using a sum of sinusoids plus noise is a powerful model for
representing audio \cite{serra1989system}, allowing for the easy implementation
of many manipulations such as time-stretching \cite{marchand2004enhanced} and
timbre-morphing \cite{haken2007beyond}.
%
Here the phase evolution of the sinusoid is assumed linear over
the analysis frame, only the phase and frequency of the sinusoids at these
analysis points are used to fit a plausible phase function after some
the analysis points are connected to form a partial
\cite{mcaulay1986speech}. 
%
Recently, there has
been interest in using higher-order phase functions \cite{xuepiecewise} as the estimation of
their parameters has been made possible by a new set of techniques of only
moderate computational complexity using signal derivatives \cite{hamilton2011non}.
%
The use
of higher-order phase models allows for accurate description of highly modulated
signals, for
example in the analysis of birdsong \cite{stowell2013improved}.
%
The frequency
modulation information has also been used in the
regularization of mathematical programs for audio source separation \cite{creager2016musicalsource}.
%
The sinusoidal model typically considered is
%
\begin{equation}
    \label{eq:polyphaseexp}
    x(t) = \exp(a_0 + \sum_{j=1}^{Q} a_j t^j) + \eta(t)
\end{equation}
where $\eta(t)$ is white gaussian noise.
%
Although this technique can be exteded to describe a single
sinusoid of arbitrary complexity simply by increasing $Q$, it is still of
interest to consider signals featuring a sum of $P$ such components, i.e.,
%
\begin{equation}
    \label{eq:polyphaseexpmix}
    s(t) = \sum_{i=1}^{P} x_{i}(t) + \eta(t)
\end{equation}
%
with
%
\[
    x_{i}(t) = \exp(a_{i,0} + \sum_{j=1}^{Q} a_{i,j} t^j)
\]
%
In situations where multiple signal sources are known to exist (e.g., recordings
of multiple speakers or performers) such a model is plausible.
%
Previous work on these methods has generally assumed signals containing a single
component, i.e., $P=1$ or applied the technique assuming the influence of other
components is negligible. Later we will refine when this assumption can be made.

\cite{hamilton2012comparisons} gives a comprehensive evaluation of the signal
derivative methods. In \cite{robel2002estimating} the extent to which two
components in mixture can corrupt estimations of the frequency slope
($\Im\{a_{0,2}\}$ and $\Im\{a_{1,2}\}$) is investigated in the context of the
reassignment method, a related technique, but the corruption of the other
parameters is not considered.

The signal-derivatives technique we will consider is the
DDM for its convenience: the technique does not
require computing derivatives of the signal to be analysed. It does, however, require a
once-differentiable analysis window.

In this paper, we evaluate the performance of various once-differentiable
windows in estimating the $a_{j}$. We are interested in windows with lower
sidelobes in order to better estimate parameters of sinusoidal chirp signals in
mixture. A technique to design once-differentiable approximations to aribitrary
symmetrical windows is presented along with a design example for a
high-performance window.

%\section{The polynomial phase sinusoid}
%Recently techniques for estimating the parameters of complex polynomial phase sinusoids,
%i.e., signals of the form
%%
%\begin{equation}
%    \label{eq:polyphaseexp}
%    x(t) = \exp(a_0 + \sum_{j=1}^{Q} a_j t^j)
%\end{equation}
%%
%with $a_{j} \in \mathbb{C}$, have been developed. Many techniques
%utilizing derivatives of the signal or the analysis window have
%been developed for estimating the $a_j$. The theoretical equivalence of these
%techniques is described in \cite{hamilton2011non}. This paper focuses on the
%Distribution Derivative Method (DDM) only \cite{betser2009sinusoidal}, as it only requires a once-differentiable
%analysis window and no signal derivatives, which is convenient for signal
%processing applications.
%
%In \cite{hamilton2012comparisons} the accuracy
%of the DDM in estimating each $a_{j}$ is evaluated for $Q=1$ with the constraint
%that $\Re{a_1}=0$ and for $2 \leq Q \leq 3$ with the Hann window used to form
%the test functions. In (Roebel) twice-differentiated windows are used to
%estimate the frequency slopes of chirp signals. Results on the accuracy of
%estimation for signals made of a mixture of chirps are given for the frequency
%slope ($\Im{a_2}$) only, but a variety of windows are compared.
%
%In this paper, we evaluate the performance of various once-differentiable
%windows in estimating the $a_{j}$. We are interested in windows with lower
%sidelobes in order to better estimate parameters of sinusoidal chirp signals in
%mixture. A technique to design once-differentiable approximations to aribitrary
%symmetrical windows is presented along with a design example for a
%high-performance window.

\section{The estimation of $a_j$}
We will now show briefly how the DDM can be used to estimate the $a_j$. Based on
the theory of distributions \cite{schwartz1959theorie}, the DDM
makes use of ``test
functions'' or atoms $\psi$. These atoms must be once differentiable with
respect to time variable $t$ and be non-zero only on a finite interval
$[-\frac{L_{t}}{2},\frac{L_{t}}{2}]$. First, we define
%
\begin{equation}
    \label{eq:ddm:inner:prod:def}
    \left\langle x , \psi \right\rangle = 
    \int_{-\infty}^{\infty}x(t)\overline{\psi}(t)dt
\end{equation}
%
and the operator 
%
\[
\mathcal{T}^{\alpha} : (\mathcal{T}^{\alpha}x)(t) = t^{\alpha}x(t)
\]
%
Consider the weighted signal
%
\[
    f(t) = x(t) \overline{\psi}(t)
\]
%
differentiating with respect to $t$ we obtain
%
\begin{multline}
    \label{eq:ddm:weighted:sig:derivative}
    \frac{df(t)}{dt} = 
    \frac{dx}{dt}(t)\overline{\psi}(t)
    + x(t)\frac{d\overline{\psi}}{dt}(t) \\
    = \left( \sum_{j=1}^{Q} j a_j t^{j-1} \right) x(t)\overline{\psi}(t)
    + x(t)\frac{d\overline{\psi}}{dt}(t)
\end{multline}
%
Because $\psi$ is zero outside of the interval $[-\frac{L_{t}}{2},\frac{L_{t}}{2}]$, integrating
$\frac{df(t)}{dt}$ we obtain
%
\begin{multline*}
    \int_{-\infty}^{\infty}\frac{df(t)}{dt}dt \\
    = \sum_{j=1}^{Q} j a_j \int_{-\frac{L_{t}}{2}}^{\frac{L_{t}}{2}} t^{j-1} x(t) \overline{\psi}(t) dt
    + \left\langle x, \frac{d\overline{\psi}}{dt} \right\rangle = 0
\end{multline*}
%
or, using the operator $\mathcal{T}^{\alpha}$,
%
\[ 
    \sum_{j=1}^{Q} j a_j 
    \left\langle \mathcal{T}^{j-1} x , \overline{\psi} \right\rangle
    = -\left\langle x, \frac{d\overline{\psi}}{dt} \right\rangle
\]
%

From this we can see that to estimate the coefficients $a_j$, $ 1 \leq q \leq Q
$ we simply need $R$ atoms with $R \geq Q$ to solve the linear system of
equations
\begin{equation}
    \label{eq:ddmsyseq}
    \sum_{j=1}^{Q} j a_j 
    \left\langle \mathcal{T}^{j-1} x , \overline{\psi}_{r} \right\rangle
    = -\left\langle x, \frac{d\overline{\psi}_{r}}{dt} \right\rangle
\end{equation}
for $1 \leq r \leq R$.

To estimate $a_0$ we write the signal we are analysing as
\[
    x(t) = \exp(a_0) \gamma(t) + \epsilon (t)
\]
where $\epsilon (t)$ is the error signal, the part of the signal that is not explained
by our model, and $\gamma (t)$ is the part of the signal
whose coefficients have already been estimated, i.e.,
\[
    \gamma(t) = \exp \left( \sum_{j=1}^{Q} a_j t^{j} \right)
\]
Computing the inner product $\left\langle x , \gamma \right\rangle$, we have
\[
    \left\langle x , \gamma \right\rangle
    =
    \left\langle \exp(a_0) \gamma , \gamma \right\rangle + 
        \left\langle \epsilon , \gamma \right\rangle
\]
The inner-product between $\epsilon$ and $\gamma$ is $0$, by the orthogonality
principle \cite[ch.~12]{kay1993fundamentals}. Furthermore, because $\exp(a_0)$ does not
depend on $t$, we have
\[
    \left\langle x , \gamma \right\rangle
    =
    \exp(a_0) \left\langle \gamma , \gamma \right\rangle
\]
so we can estimate $a_0$ as
\begin{equation}
    \label{eq:ddmesta0}
    a_0 = \log \left( \left\langle x , \gamma \right\rangle \right)
        - \log \left( \left\langle \gamma , \gamma \right\rangle \right)
\end{equation}

\section{The estimation of the $a_{i,j}$ of $P$ components}

We examine how the mixture model influences the estimation of the $a_{i,j}$ in
Eq.~\ref{eq:polyphaseexpmix}.
Consider a mixture of $P$ components.
If we define the weighted signal sum
%
\[
    g(t) = \sum_{i=1}^{P} x_{i}(t) \overline{\psi}(t) = \sum_{i=1}^{P} f_{i}(t)
\]
%
and substitute $g$ for $f$ in Eq.~\ref{eq:ddm:weighted:sig:derivative} we obtain
%
\begin{multline}
    \label{eq:mixest}
    \sum_{i=1}^{P} \int_{-\frac{L_{t}}{2}}^{\frac{L_{t}}{2}} \frac{df_{i}}{dt}(t) =
    0
    \\ = 
    \sum_{i=1}^{P}
    \sum_{j=1}^{Q} j a_{i,j} 
    \left\langle \mathcal{T}^{j-1} x_i , \overline{\psi} \right\rangle
     + \left\langle x_i, \frac{d\overline{\psi}}{dt} \right\rangle
\end{multline}
%
From this we see if $\left\langle \mathcal{T}^{j-1} x_i , \overline{\psi}_{r}
\right\rangle$ and $\left\langle x_i, \frac{d\overline{\psi_{r}}}{dt} \right\rangle$
are small for all but $i = i^{\ast}$ and a subset of $R$ atoms, we
can simply estimate the parameters $a_{i^{\ast},j}$ using
\[
    \sum_{j=1}^{Q} j a_{{i^{\ast}},j} 
    \left\langle \mathcal{T}^{j-1} x_{i^{\ast}} , \overline{\psi}_{r} \right\rangle
    = -\left\langle x_{i^{\ast}}, \frac{d\overline{\psi}_{r}}{dn} \right\rangle
\]
for $1 \leq r \leq R$. To compute $a_{i^{\ast},0}$ we simply use
\[
    \gamma_{i^{\ast}}(t) = \exp \left( \sum_{j=1}^{Q} a_{i^{\ast},j} t^{j} \right)
\]
in place of $\gamma$ in Eq.~\ref{eq:ddmesta0}.

\section{Designing the $\psi_{r}$}
\label{sec:designingatoms}
%
\begin{figure*}[t]
    \centerline{\includegraphics[width=7in]{{search_dpw_bw_m}.eps}}
\caption{\label{fig:dpw} Comparing the main-lobe and asymptotic power
spectrum characteristics of the continuous 4-term Nuttall window, the digital
prolate window with $W=0.008$, and the continuous approximation to the digital
prolate window.}
\end{figure*}
%
In practice, an approximation of Eq.~\ref{eq:ddm:inner:prod:def} is evaluated using
the discrete Fourier transform (DFT) on a signal $x$ that is properly sampled
and so can be evaluated at a finite number of points $nT$ with $n \in [0,N-1]$ and
$T$ the sample period in seconds. In this way, the atoms $\psi_{\omega}(t)$ chosen
are the products of the elements of the Fourier basis and an appropriately
chosen window $w$ that is once differentiable and finite, i.e.,
%
\[
    \psi_{\omega}(t) = w(t) \exp(-\sqrt{-1} \omega t)
\]
%
Defining $N = \frac{L_{t}}{T}$ and $\omega = 2
\pi \frac{r}{N}$, the approximate
inner product is then
%
\[
    \left\langle x , \psi_{\omega} \right\rangle \approx 
    \sum_{n=0}^{N-1} x(Tn) w(Tn) \exp(-2 \pi \sqrt{-1} r \frac{n}{N}) 
\]
%
i.e., the definition of the DFT of a windowed signal.%
\footnote{%
    Notice however that this is an approximation of the inner product and should
    not be interpreted as yielding the Fourier series coefficients of a properly
    sampled signal $x$ periodic in $L_{t}$. This means that other
    evaluations of the inner product that yield more accurate results are
    possible. For example, the analytic solution is possible if $x$ is assumed
    zero outside of $[-\frac{L_{t}}{2},\frac{L_{t}}{2}]$ (the $\psi$ are in
    general analytic).  In this case the samples of $x$ are convolved with the
    appropriate interpolating sinc functions and the integral of this function's
    product with $\psi$ is evaluated.
}%
The DFT is readily interpreted as a bank of bandpass filters centred at
normalized frequencies%
\footnote{%
In this article as in much of the literature on signal processing the index $r$
of the Fourier basis element corresponding to the frequency $\omega_{r} = 2 \pi
\frac{r}{N}$ will be referred to as its bin number.
}
$\frac{r}{N}$ and with frequency response described by the Fourier transform of
modulated $w$ \cite{allen1977unified}. Therefore choosing $\psi$ amounts to a
filter design problem under the constraints that the impulse response of the
filter be differentiable in $t$ and finite. To minimize the influence of all but
one component, granted the components's energy concentrations are sufficiently
separated in frequency, we desire impulse responses whose magnitude response
gives maximum out-of-band rejection or equivalently, windows whose Fourier
transform exhibits the lowest sidelobes.

In all the publications these authors reviewed on the DDM, the window used was
the Hann window which is once-differentiable everywhere in the time-domain. In
\cite{robel2002estimating}, a publication on the reassignment method, other
windows than the Hann are considered but these windows must be
twice-differentiable.  Nuttall \cite{nuttall1981some} has designed windows with
lower sidelobes than the canonical Hann window which are everywhere at least
once-differentiable. It is also possible to design approximations to aribtrary
symmetrical window functions using harmonically related cosines, as is discussed
in the following section.

\section{Differentiable approximations to windows}
%
\begin{figure*}[ht]
    \centerline{\includegraphics[width=7in]{{ddm_snr_win_comp}.eps}}
\caption{\label{fig:snrwincomp} The estimation variance of random polynomial
    phase sinusoids averaged over $K_{1}=\Ksnr{}$ trials using atoms generated from
    various windows. \textit{C} is the Cram\'{e}r-Rao
lower bound, \textit{N3} and \textit{N4} are the 3- and 4-cosine-term continuous
Nuttall windows, \textit{H} is the Hann window, and \textit{P5} is the continous
5-cosine-term approximation to a digital prolate window as described in
Sec.~\ref{sec:designexample}.}
\end{figure*}
%
A differentiable approximation to a symmetrical window can be designed in a
straightforward way. In \cite{harris1978use} and \cite{rabiner1970approach} it
is shown how to design optimal windows of length $N$ samples using a linear
combination of $M$ harmonically related cosines
\begin{equation}
    w(n) = \sum_{k=0}^{M-1} b_{k} \cos (2 \pi k \frac{n}{N})
\mathcal{R}(\frac{n}{N})
\end{equation}
where $\mathcal{R}$ is the \textit{rectangle function}. This function is
discontinuous
at $n = \frac{\pm N}{2}$, and therefore not differentiable there, unless
\[
\sum_{k=0}^{M-1} b_{k} \cos ( \pm \pi k ) = 0
\]

Rather design based on an optimality criterion, such as the height of the
highest sidelobe, a once-differentiable approximation to an existing window $v$ is
desired. To do this, we choose the $b_{k}$ so that the window $w$'s squared approximation error to
$v$ is minimized
while having $w(\frac{\pm N}{2}) = 0$, i.e. we find the solution $\{
b^{\ast}_{k} \}$ to the mathematical program
\begin{equation}
    \label{eq:searchcontwinprogram}
    \text{minimize}
    \sum_{n=0}^{N-1} ( v(n) 
        - \sum_{k=0}^{M-1} b_{k} \cos(2 \pi k \frac{n}{N}))^{2}
\end{equation}
\[
    \text{subject to} \\
    \sum_{k=0}^{M-1} b_{k} \cos(\pi k ) = 0
\]
which can be solved using constrained least-squares -- a standard numerical
linear algebra routine \cite[p.~585]{golub1996matrix}.

\section{A continuous window design example}
\label{sec:designexample}
%
\begin{figure*}[ht]
    \centerline{\includegraphics[width=7in]{{comp_offset_chirp_est_err_params}.eps}}
\caption{\label{fig:mixwincomp} The mean squared estimation error for each
    parameter in an analysis of two components in mixture. A set of
    $K_{2}=\Koffset{}$ chirps was synthesized and each unique pair used for
    maximum bin differences $0 \leq d < \Doffset{}$. The signal power ratio
    between components is indicated using shades of grey: lighter shades
    indicate a ratio closer to $1$. The ratio in decibels is indicated in the
    plot legend. The names indicate the windows used to generate the atoms for
    estimation: \textit{N3} and \textit{N4} are the 3- and 4-cosine-term
    continuous Nuttall windows, \textit{H} is the Hann window, and \textit{P5}
is the continous 5-cosine-term approximation to a digital prolate window as
described in Sec.~\ref{sec:designexample}.}
\end{figure*}
%
As a design example we show how to create a continuous approximation of a digital
prolate spheroidal window.

Digital prolate spheroidal windows are a parametric approximation to functions
whose Fourier transform's energy is maximized in a given bandwidth
\cite{slepian1978prolate}. These can be tuned to have extremely low sidelobes,
at the expense of main-lobe width.  Differentiation of these window functions
may be possible but is not as straightforward as differentiation of the
sum-of-cosine windows above. Furthermore, the windows do not generally have
end-points equal to 0. In the following we will demonstrate how to approximate
a digital prolate spheroidal window with one that is everywhere at least
once-differentiable.

In \cite{verma1996digital} it was shown how to construct digital prolate
spheroidal windows under parameters $N$, the window length in samples, and a
parameter $W$. We chose $N=512$ based on the window length chosen in
\cite{betser2009sinusoidal} for ease of comparison. Its $W$ parameter's value
was chosen by synthesizing windows with $W$ ranging between $0.005$ and $0.010$
at a resolution of $0.001$. The window with the closest 3db bandwidth to the
4-term Nuttall window was obtained with $W=0.008$. Its magnitude response is
shown in Fig.~\ref{fig:dpw}. We see that
this window's asymptotic falloff is 6 dB per octave and therefore has a
discontinuity somewhere in its domain \cite{nuttall1981some}. We designed an approximate
window using Eq.~\ref{eq:searchcontwinprogram} for $M$ varying between $2$ and
$N/8$ to find the best approximation to the digital prolate window's main lobe
using a small number of cosines. The $M$ giving the best approximation was $5$.
The magnitude response of the approximation is shown in
Fig.~\ref{fig:dpw}.  We see that we obtain a lower highest sidelobe
level than the Nuttall and Prolate windows by slightly sarificing the narrowness
of the main lobe. Furthermore in Fig.~\ref{fig:dpw} we observe that
the falloff of the window is 18 dB per octave because it is once-differentiable
at all points in its domain.

\section{The performance of improved windows}

To compare the average estimation error variance with the
theoretical minimum given by the Cram\'{e}r-Rao bound we synthesized $K_{1}$ random
chirps using Eq.~\ref{eq:polyphaseexp}
with $Q=2$ and parameters choosen from uniform distributions justified in
\cite{betser2009sinusoidal}. The original Hann window,
the windows proposed by Nuttall and the new digital prolate based window were
used to synthesize the atoms as described in Sec.~\ref{sec:designingatoms} and
their estimation error variance was compared (see
Fig.~\ref{fig:snrwincomp}). After performing the DFT to obtain inner products
with the atoms, the three atoms whose inner products were greatest were used in
the estimations, i.e., $R=3$ in Eq.~\ref{eq:ddmsyseq}. We see that windows with
lower sidelobes exhibit the lowest possible error variance at the sacrifice of
some estimator efficiency.

To evaluate the performance of the various windows when estimating the
parameters of components in mixture we synthesized signals using
Eq.~\ref{eq:polyphaseexpmix} with $P=2$ and $Q=2$ and parameters chosen from the
uniform distributions specified in \cite{betser2009sinusoidal}. A set
$\mathcal{C}$ of $K_{2}$ components was synthesized. The atom $r^{\ast}$ for which
the inner product was maximized was determined for each unmixed chirp and the
chirp was modulated by $\exp(-2\pi \frac{r^{\ast} n}{N}\sqrt{-1})$ for $0 \leq n
< N$ in order to move this maximum to $r=0$. Then for each desired difference
$d$ between maxima in bins, with $0 \leq d < D$, two unique chirps were
selected from $\mathcal{C}$ and one chirp was modulated by $\exp(2\pi
\frac{d n}{N}\sqrt{-1})$ for $0 \leq n < N$ in order to give the desired difference in
maxima. This component was also scaled by a constant to give a desired signal
power ratio with the other component. The parameters of both components were
estimated using the bins surrounding $r=0$ and $r=d$ for each component
respectively and the squared estimation error for each parameter was summed and divided
by $K_{2}(K_{2}-1)$ to give the averaged squared estimation error for each
parameter at each difference $d$.

In Fig.~\ref{fig:mixwincomp} we see that there is generally less estimation
error for components having similar signal power. This is to be expected as
there will be less masking of the weaker signal in these scenarios.  The
estimation error is large when the atoms containing the most signal energy for
each component are not greatly separated in frequency. This is due to the
convolution of the Fourier transform of the window with the signal, and agrees
with what was predicted by Eq.~\ref{eq:mixest}: indeed windows with a larger
main lobe exhibit a larger `radius' in which the error of the parameter
estimation will be high.  However, for signals where local inner product maxima
are from atoms sufficiently separated in frequency, windows with lower sidelobes
are better at attenuating the other component and for these the estimation error
is lowest.

\section{Conclusions}

Motivated by the need to analyse mixtures of frequency- and
am\-pli\-tude-mod\-u\-lat\-ed
sinusoids (Eq.\ref{eq:polyphaseexpmix}), we have shown that the DDM can be
employed under a single-component
assumption when components have roughly disjoint sets of atoms for which their
inner products take on large values. This indicates the need for windows whose
Fourier transform exhibits low sidelobes. We developed windows whose sidelobes
are minimized while remaining everywhere once-differentiable: a requirement to
generate valid atoms for the DDM. These windows were shown to improve parameter
estimation of $P=1$ and $P=2$ components in mixture, granted the components
exhibited reasonable separation in frequency between the atoms for which the
inner product was maximized.

Further work should evaluate these windows on sinusoids of different orders,
i.e., $Q \gg 1$. Optimal main-lobe widths for windows should be determined
depending on the separation of local maxima in the power spectrum. It should
also be determined if these windows improve the modeling of real-world acoustic
signals.

%\section{Acknowledgments}

%\newpage
%\nocite{*}
\bibliographystyle{IEEEbib}
\bibliography{paper} % requires file DAFx17_tmpl.bib

%\section{Appendix}

\end{document}
